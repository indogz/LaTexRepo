\documentclass[10pt]{report}
\usepackage[lmargin=1.5cm, rmargin=1.5cm, tmargin=1cm, bmargin=1cm]{geometry}
\usepackage[utf8]{inputenc}
\usepackage{graphicx}
\usepackage{verse}
\usepackage[bookmarks]{hyperref}



\begin{document}
	
	\title{Riassunti di Italiano}
	\author{Matteo Incani}	
	\maketitle
	\twocolumn
	\chapter*{Decadentismo}
	
		\section{Il movimento}
		
	Nel 1883 Paul Verlaine, a Parigi, pubblicò un sonetto nel periodico “Le Chat Noir” in cui affermava di identificarsi con l’atmosfera di estenuazione dell’impero romano alla fine della decadenza.
	 Il sonetto interpretava uno stato d’animo diffuso nella cultura del tempo, il senso di disfacimento di una civiltà. Queste idee erano proprie dei circoli d’avanguardia e si contrapponevano alla mentalità borghese.
	  Il termine “decadentismo” fu usato dalla critica ufficiale in accezione negativa, ma i gruppi intellettuali la vollero assumere in forma di protesta. 
	  
	Inizialmente la parola “decadentismo” indicava un determinato movimento letterario, successivamente fu esteso a contesti più vasti e fu usato come etichetta di una grande corrente culturale. Il Decadentismo appare in contesti artistici e letterari anche molto diversi tra loro.
	
	Il movimento rifiuta la visione positivistica, quindi la mentalità borghese, ed il decadente ritiene che la scienza e la ragione non possono costituire la vera conoscenza del reale; solo rinunciando alla visione razionale si può cogliere il vero significato delle cose.
	 La scoperta dell’inconscio è il dato fondamentale della cultura decadente, è il luogo in cui si mescola l’identità dell’io e del mondo. Se io e mondo quindi non sono in realtà distinti, l’io può annullarsi nella vita del gran Tutto (farsi nuvola, filo d’erba, corso d’acqua) e potenziare la propria vita all’infinito, quest’atteggiamento viene chiamato panismo.
	 
	 \section[Estetismo]{L'estetismo}

	 L’esteta è l’uomo che vuole trasformare la sua vita in opera d’arte, ossia il suo valore più grande.
	  L’esteta è colui che assume come principio di vita solamente il bello, egli si colloca oltre la morale comune, e cerca continuamente sensazioni rare e squisite.
	   Il poeta quindi rifiuta di farsi banditore di idee morali e civili, così da poter creare una poesia pura (non influenzata).
		
		\section{Il superuomo di Nietzsche}
		Secondo Nietzsche l’uomo moderno si è lasciato imprigionare la mente dalla borghesia, dal suo modo di pensare, e soprattutto dalla Chiesa.
		 La soluzione per ottenere una vita originale e creativa è elevarsi alla condizione di oltreuomo (contrapposto al subuomo), ovvero di abbandonare la morale tradizionale e di non farsi influenzare, mirando ad uno sviluppo estremo dell’individualità. 
		 Successivamente il concetto di superuomo sarà reinterpretato e stravolto da\\ Gabriele d’Annunzio, arrivando addirittura all’appropriazione che ne fece il nazismo.
		\section{Pascoli}
		
		\subsection{La vita}
	
		Giovanni Pascoli nacque a San Mauro di Romagna da una famiglia della piccola borghesia rurale. La sua infanzia venne segnata in modo indelebile dalla morte del padre, ucciso mentre rincasava, nonchè da una serie di lutti familiari.
		
		Trasferitosi a Rimini, si diplomò e grazie ad una borsa di studio conferitogli da una commissione in cui era presente anche Carducci, Pascoli iniziò il percorso universitario a Bologna.
		
		La lunga serie di lutti che segna la vita di Pascoli provoca nel poeta dei traumi psicologici che il poeta contrasta chiudendosi in un nido familiare insieme alle sorelle rimaste che ebbero per lui una raffigurazione materna.
		Non ebbe mai relazioni amorose e condusse una vita forzatamente casta, le esigenze affettive venivano soddisfatte dalle sorelle.
		Ne deriva un'indole \textbf{ depressiva e turbata}.
		
		La formazione di Pascoli fu positivistica com'era in voga all'epoca degli studi universitari e per questo motivo nelle sue opere troviamo dettagli di ornitologia e botanica, nonchè temi astrali.
		In Pascoli notiamo una crisi verso la scienza che si trasforma nell'attrazione verso l'ignoto, caratteristica tipica del decadentismo.
		
		\subsection[Fanciullino]{Il fanciullino}
		Nella produzione Pascoliana vi è un saggio, "Il fanciullino" in cui spiega che in ogni uomo sopravvive in fondo un fanciullino che vede tutte le cose con ingenuo stupore.
		Utilizza un linguaggio che si spinge nell'intimo delle cose per riscoprirle nella loro freschezza originaria.
		Dietro alla metafora del fanciullino si nota la poesia come conoscenza fantasiosa.
		Il poeta fanciullino grazie al suo modo alogico, fa sprofondare nell'abisso della verità e della conoscenza profonda della realtà, l'essenza segreta delle cose, appare quindi come un veggente.
		
		\subparagraph[Simbolismo]{Il simbolismo}Pascoli ricevette una formazione classica e positivistica. A cavallo tra i due secoli però in Pascoli si riflette il nuovo affermarsi di idee spiritualistiche e idealistiche. La crisi positivistica si riflette sulla poetica di Pascoli; gli oggetti materiali hanno un rilievo fortissimo, e vengono filtrati attraverso la visione soggettiva del reale che li caricano di valenze simboliche. Questa soggettivazione del reale dà spazio a legami non razionali tra e cose, che trasportano direttamente al significato senza passare per un filo logico.

		\newpage
		\subsection[Poesie]{Poesie}
		
		
		\renewcommand{\poemtoc}{subsection}
		\poemtitle{X Agosto}
		\settowidth{\versewidth}{There was an old party of Lyme}
		
		\begin{verse}[\versewidth]
			San Lorenzo , io lo so perché tanto\\
			di stelle per l'aria tranquilla\\
			arde e cade, perché si gran pianto\\
			nel concavo cielo sfavilla.\\
			
			
			Ritornava una rondine al tetto :\\
			l'uccisero: cadde tra i spini;\\
			ella aveva nel becco un insetto:\\
			la cena dei suoi rondinini.\\
			
			
			Ora è là, come in croce, che tende\\
			quel verme a quel cielo lontano;\\
			e il suo nido è nell'ombra, che attende,\\
			che pigola sempre più piano.\\
			
			
			Anche un uomo tornava al suo nido:\\
			l'uccisero: disse: Perdono ;\\
			e restò negli aperti occhi un grido: \\
			portava due bambole in dono.\\
			
			
			Ora là, nella casa romita,\\
			lo aspettano, aspettano in vano:\\
			egli immobile, attonito, addita\\
			le bambole al cielo lontano.\\
			
			
			E tu, Cielo, dall'alto dei mondi\\
			sereni, infinito, immortale,\\
			oh! d'un pianto di stelle lo inondi\\
			quest'atomo opaco del Male!\\
		\end{verse}
	
		\subparagraph[X Agosto]{Spiegazione}Cambia la lingua, toglie le parole arcaiche.
		E' una poesia drammatica che porta analogie con il mondo animale. 
		Uso frequente della punteggiatura e delle figure retoriche.
		\textbf{Parallelismo} (rondine uomo)e \textbf{onomatopea} ("pigola" v12)
		
		

		
		\renewcommand{\poemtoc}{subsection}
		\poemtitle{Temporale}
		\settowidth{\versewidth}{There was an old party of Lyme}
		
		\begin{verse}[\versewidth]
				Un bubbolìo lontano. . .\\			
			
			Rosseggia l’orizzonte,\\			
			come affocato, a mare:\\		
			nero di pece, a monte,\\			
			stracci di nubi chiare:\\			
			tra il nero un casolare:\\			
			un’ala di gabbiano.\\
			
		\end{verse}
		
		\subparagraph[Temporale]{Spiegazione}
		Tanti indizi ci dicono che il poeta sta comunicando uno stato d'animo tormentato, di cui la tempesta e i  colori sono “il simbolo”. Ce lo dicono le parole che ha scelto (rosso "affocato", dal nero "di pece", dagli "stracci" di nubi); ce lo dice, anche  l’utilizzo di brevi frasi senza verbo, poste una dopo l'altra, che non lasciano spazio ai dettagli nella descrizione della natura e sembrano esprimere direttamente uno stato d'animo sbigottito.
		
		Analogia: casolare->ala di gabbiano
		Onomatopea: bubbolio
		
		\renewcommand{\poemtoc}{subsection}
		\poemtitle{Novembre}
		\settowidth{\versewidth}{There was an old party of Lyme}
		
		\begin{verse}[\versewidth]
			Gemmea l'aria, il sole così chiaro\\
			che tu ricerchi gli albicocchi in fiore,\\
			e del prunalbo l'odorino amaro\\
			senti nel cuore...\\
			Ma secco è il pruno, e le stecchite piante\\
			di nere trame segnano il sereno,\\
			e vuoto il cielo, e cavo al piè sonante\\
			sembra il terreno.\\
			Silenzio, intorno: solo, alle ventate,\\
			odi lontano, da giardini ed orti,\\
			di foglie un cader fragile. E' l'estate\\
			fredda, dei morti.\\
			
		\end{verse}
		
		\subparagraph[Novembre]{Spiegazione}Ricorre il tema della morte.
		\textbf{Sinestesia} (odorino – amaro), \textbf{allitterazione} (segnano il sereno), \textbf{ossimoro} (estate – fredda) .
	
		
		
		\renewcommand{\poemtoc}{subsection}
		\poemtitle{Lampo}
		\settowidth{\versewidth}{There was an old party of Lyme}
		
		\begin{verse}[\versewidth]
			E cielo e terra si mostrò qual era:\\
			la terra ansante, livida, in sussulto;\\
			il cielo ingombro, tragico, disfatto:\\
			bianca bianca nel tacito tumulto\\
			una casa apparì sparì d'un tratto;\\
			come un occhio, che, largo, esterrefatto,\\
			s'aprì si chiuse, nella notte nera.\\
			
		\end{verse}
		
		\subparagraph[Lampo]{Spiegazione}Descrizione di un trauma improvviso. Non si sente nulla ma vi è confusione. Nei versi 4 e 5 dovrebbero esserci virgole ma non ci sono mentre nel verso 6 ne mette quattro di fila.
		
		\textbf{Ossimoro}: tacito tumulto
		\textbf{Alliterazione}: tacito tumulto
		
		\renewcommand{\poemtoc}{subsection}
		\poemtitle{La mia sera}
		\settowidth{\versewidth}{There was an old party of Lyme}
		
		\begin{verse}[\versewidth]
			[...]
			
			Don ... Don ... E mi dicono, Dormi!\\
			mi cantano, Dormi! sussurrano,\\
			Dormi! bisbigliano, Dormi!\\
			là, voci di tenebra azzurra ...\\
			Mi sembrano canti di culla,\\
			che fanno ch'io torni com'era ...\\
			sentivo mia madre ... poi nulla ...\\
			sul far della sera\\
			
		\end{verse}
		
		\subparagraph[La mia sera]{Spiegazione}La mia sera, contenuta nella raccolta I canti di Castelvecchio, racconta di una sera dopo un temporale. Il poeta contempla lo spettacolo della natura rasserenata e rinfrescata dal temporale e in cui pullulano mille vite canore. Per analogia confronta la vicenda naturale con la propria vita, contrassegnata da dolori e lutti, che sembra aver finalmente trovato un po’ di pace. Egli si sente in armonia e si domanda che ne sono dei dolori e delle acerbità del passato. Tutto viene ricondotto al tema del nido, dell’infanzia, caro al Poeta. Il nido, visto come centro di affetti ed di emozioni intime, il legame con la madre, isolano dall’esterno e creano un’atmosfera rassicurante e protettiva. 
		Viene vista sia sul piano naturalistico come parte del giorno sia sul piano simbolico come parte della vita.
		
		\textbf{Climax discendente}: don don...

		
		
		\section{D'annunzio}
			\subsection[Vita]{La vita}
			D’Annunzio nacque nel 1863 a Pescara da famiglia borghese. Egli studiò in una delle scuole più aristocratiche d’Italia. A 16 anni pubblicò la sua prima raccolta di liriche. A 18 anni si trasferì a Roma per studiare all’università, idea che abbandonò molto presto, preferendo vivere tra salotti mondani e redazioni di giornali. Esercitò per un periodo quindi la professione di giornalista pubblicando articoli soprattutto sulla “Tribuna” di Roma. Negli anni ’90 si trasferì a Napoli per sfuggire ai creditori, dove scrisse per “Il Mattino”. Quegli anni furono il periodo in cui D’Annunzio si creò la maschera dell’esteta, non solo attraverso le opere, ma soprattutto attraverso la vita, messa in mostra e fatta di continue avventure galanti, di lusso, di duelli. Egli in quanto esteta rifiuta la mediocrità borghese e si rifugia in un mondo fatto di pura arte, che disprezza la morale corrente. 
			Nel 1897 cominciò l’avventura parlamentare, seguendo i principi da lui rielaborati del superuomo e del poeta vate, come deputato di estrema destra. Successivamente per seguire la “vita” passò ad un orientamento di sinistra. A causa dei creditori scappò in Francia nel 1910 dove però non interruppe i legami con la propria patria.
			Si ricorda inoltre che D’Annunzio partecipò attivamente alla prima guerra mondiale, sostenne un’intensa campagna interventista e compì diverse imprese come il volo su Vienna. In seguito alla conclusione del conflitto, lo scrittore strinse contatti con il fascismo, che lo dipingeva come “padre della patria”, pur guardandolo con sospetto e confinandolo in una villa di Gardone, dove morì nel 1938.
			\subsection[Esteta]{D'annunzio esteta}
			L'esteta è, per D'Annunzio, colui che cerca di vivere la propria vita come un'opera d'arte, ed egli stesso si pose quest'obiettivo, di cui sono testimonianza le vicende autobiografiche dei protagonisti dei suoi romanzi. In tal modo l'estetismo, più che una formulazione teorica, diventa un vero e proprio stile di vita come descritto nel romanzo "\textit{Il piacere}", ricco di elementi autobiografici.
			\subsection[Superuomo]{Il superuomo}
			Durante gli anni ’90 la fase estetizzante di D’annunzio attraversò una crisi; per superarla trovò la soluzione nel mito del superuomo, creato precedentemente da Nietzsche.\\
			 D’Annunzio quindi elaborò e stravolse il concetto, introducendovi il “vivere inimitabile”, ovvero il poeta doveva creare l’immagine di una vita eccezionale, non solo con le opere, ma anche attraverso le imprese incredibili che mettevano in mostra appunto la vita dell’individuo. Egli sosteneva inoltre che il poeta (quindi il superuomo) dovesse aprire la strada ad una nuova élite, di pochi esseri eccezionali, atta a guidare il popolo e a porre fine al liberalismo borghese alla democrazia e all’egualitarismo, dando così al poeta un ruolo di vate.
			
			
			\subsection[Vate]{Il poeta vate}
			La figura del poeta vate è attribuita agli autori che cercano di interpretare e guidare i sentimenti delle masse di ogni epoca.
			Il rapporto con il fascismo fu complesso e articolato, benché sostanzialmente organico: i fascisti in ascesa celebrarono D'Annunzio, riutilizzando i motti e i simboli del Vate già utilizzati a Fiume, come uno dei massimi e più fecondi letterati d'Italia, ma lo scrittore, dopo l'adesione iniziale ai Fasci di combattimento, non prese mai la tessera del Partito Nazionale Fascista, probabilmente per mantenere la sua completa autonomia.
			
			Si rese protagonista di azioni eroiche come il \textbf{Volo su Vienna} e la partecipazione al primo conflitto mondiale anche nella X MAS a Venezia. 
			D'Annunzio, che fu nell'equipaggio dei tre MAS che effettuarono la \textbf{Beffa di Buccari}, ebbe sempre una particolare simpatia per il nascente gruppo degli incursori della Marina e la sua influenza a livello politico gli consentì di propugnarne a più riprese il potenziamento.
			
			\subsection[Panismo]{Il panismo}
			Il Panismo (detto anche sentimento panico della natura) è una percezione molto profonda del mondo esterno (soprattutto paesaggi naturali) che crea una fusione tra l'elemento naturale e quello più specificatamente umano. E' la tensione a identificarsi con le forze naturali e a fondersi con esse istintivamente.
			
			Quello di D'Annunzio consiste nel considerare la natura come un'entità viva e movimento continuo.
			Con questa entità l'uomo deve fondersi e stabilire un contatto intenso, fino ad immergersi nel suo ritmo vitale; uomo e mondo si uniscono e entrano direttamente in contatto. 
			
			
			Egli cerca una fusione dei sensi e dell'animo con le forze della vita, accogliendo in sé e rivivendo l'esistenza molteplice della natura, con piena adesione fisica, prima ancora che spirituale. E' questo il "panismo dannunziano", quel sentimento di unione con il tutto, che ritroviamo in tutte le poesie più belle di D'Annunzio, in cui riesce ad aderire con tutti i sensi e con tutta la sua vitalità alla natura, s'immerge in essa e si confonde con questa stessa.
			
			Esempio classico di metamorfosi panica: "la pioggia nel pineto" in cui si compie la completa fusione della donna (Ermione) con la natura.
			
			\subsection[Opere]{Opere}
			Durante l'anno abbiamo trattato la poesia "\textbf{La piogga nel pineto}" pubblicata nel libro \textbf{Alcyone} facente parte della raccolta delle "\textbf{Laudi}" insieme ai libri "Maia" ed "Elettra"
			Nella fase estetizzante D'Annunzio scrisse inoltre il romanzo "\textbf{Il piacere}"
	
	\chapter*{Romanzo psicologico}
	
	\section{Pirandello}
		\subsection{Vita}
		Luigi Pirandello nasce nel 1867 ad Agrigento da una famiglia borghese. Frequenta diverse università in Italia, per poi concludere gli studi accademici a Bonn, città in cui ha l’opportunità di entrare in contatto con gli autori tedeschi, fondamentali per le sue teorie sull’umorismo.
		Successivamente sin trasferisce a Roma, dove si dedica completamente all’
		insegnamento e alla letteratura, pubblicando diverse opere. 
		Come Svevo, Pirandello vive l’esperienza della declassazione a causa di un allagamento alla miniera su cui il padre aveva investito tutto. Per arginare il problema, lo scrittore intensificò notevolmente la sua produzione di novelle (1904 -1905). Dal 1915 Pirandello cominciò a dedicarsi principalmente alle opere teatrali, si ricordano in particolare: Lumìe di Sicilia, Enrico IV, Così è (se vi pare), Sei personaggi in cerca d’autore.
		Pirandello era sostanzialmente un interventista e subito dopo il delitto Matteotti si iscrisse al partito fascista per ottenere l’appoggio del regime in ambito teatrale, ma se ne allontana qualche anno dopo, riconoscendo nel fascismo una di quelle istituzioni sociali che impongono delle “maschere”. Nel 1934 vince il Nobel per la letteratura e il teatro; muore a Roma nel 1936.
		
		\subsection{Il ruolo della psicoanalisi}
		
		Alla base della visione del mondo vi è una concezione vitalistica, ossia la realtà tutta è “\textbf{vita}”, “flusso continuo, incandescente, indistinto”, qualsiasi cosa si stacchi da essa assume “forma” distinta e individuale, cioè comincia a “morire”. 
		Così fa l’uomo, che tende a cristallizzarsi in una forma individuale, una personalità coerente e unitaria, che egli stesso si dà. Non solo\textbf{ ognuno di noi si crea una “forma”}, ma chiunque attorno a noi ce ne attribuisce diverse, ciascuno secondo la propria realtà.
		Ognuna di queste “forme” è una \textbf{costruzione fittizia, una “maschera”} che noi stessi ci imponiamo e che ci impone il contesto sociale, sotto la quale non vi è “nessuno”. 
		La teoria della \textbf{frantumazione dell’io}, secondo cui l’io appunto perde la sua identità, si frantuma in una serie di stati incoerenti, è causa soprattutto delle trasformazioni della società dell’epoca (società di massa), degli apparati burocratici e dell'industrializzazione. 
		Viceversa l’individuo soffre anche ad essere fissato nelle \textbf{forme che gli vengono attribuite}.
		Le due maggiori “\textbf{trappole}” che imprigionano l’individuo nelle “forme” sono la \textbf{società} e, quella per eccellenza, la \textbf{famiglia}. 
		\textbf{In una prima fase questi processi inducono a rifiutare la realtà oggettiva e a chiudersi nella soggettività ma poi questa finirà progressivamente per sfaldarsi}.
		Secondo Pirandello non ci sono vie d’uscita dalle “trappole”, l\textbf{’unica salvezza è la fuga nell’irrazionale, o nell’ironia,} che permette di contemplare la realtà da una prospettiva straniata.
		Inoltre Pirandello formula la visione del relativismo conoscitivo, non dà una verità oggettiva fissata a priori, ma ognuno ha la sua verità che nasce dal suo modo soggettivo di vedere le cose.
		Infine si ricorda il concetto di umorismo adottato dallo scrittore;\textbf{ il tragico e il comico vanno sempre insieme}, in quanto se ci si sofferma su una banalità che inizialmente fa ridere, si può scorgere il carattere molteplice e tragico che si cela dietro (mostriamo comprensione).
		\subsection{Il fu Mattia pascal}
			\subsubsection[Trama]{La trama}
			Il fu Mattia Pascal è il più famoso dei romanzi di Pirandello. Mattia Pascal apparteneva a una famiglia benestante ma, alla morte del padre, un amministratore disonesto si era arricchito alle spalle della stessa sino a ridurla sul lastrico. Il protagonista per vivere fa il bibliotecario ma, a causa di profondi dissidi con la moglie e con la suocera, decide di partire per Montecarlo; lì la fortuna gli sorride e, fatto un piccolo capitale, decide di tornare a casa. Sul treno, mentre sfoglia il giornale per ingannare il tempo, legge di un suicidio nel suo paese e, tra lo stupore e la sorpresa, apprende che si tratta del "suo" suicidio. 
			Nella gora di un mulino era stato trovato un corpo in avanzato stato di putrefazione che era stato prontamente riconosciuto come suo dalla moglie e dalla suocera. Allora pensa di inventare un'identità e vivere una nuova vita riacquistando la libertà che aveva  	sposandosi; comincia così la vita di Adriano Meis. Affitta una casa a Roma ma dopo poco si rende conto che, senza documenti, non può vivere da persona normale, si innamora della figlia del 	padrone di casa e non la può sposare, subisce un furto e non lo può denunziare pur conoscendo il 	colpevole, viene insultato e non può vendicare l'offesa, decide allora la seconda morte di Adriano 	lasciando su un ponte il cappello, il bastone e un biglietto di riconoscimento.Tornato in paese però si accorge che non c'è più posto per Mattia Pascal, la moglie si 	è risposata ed ha avuto dei figli e lui viene trattato con diffidenza. Decide allora di 	vivere nella biblioteca con il collega, unico amico che gli sia rimasto, e, di tanto in tanto, porta fiori 	alla propria tomba "50 anni dopo la mia terza, ultima e definitiva morte". \textbf{Romanzo Psicologico: 1 persona, evoluzione mentale, digressioni.}
			
		\subsection{Novelle}
	
	\section{Svevo}
		\subsection{Vita}
		Aron Hector Schmitz nacque nel 1861 a Trieste da una famiglia borghese e orientamento religioso ebraico. Essendo Trieste sotto il dominio asburgico egli fu influenzato dalle culture: italiana, tedesca e slava; permettendogli così una prospettiva più ampia rispetto a molti scrittori italiani dell’epoca. Gli studi adolescenziali di Svevo si divisero fra Germania e Trieste, ricevendo un’istruzione in campo commerciale, per seguire le orme del padre. In Germania nacque la sua passione per la letteratura grazie ad alcuni scrittori tedeschi. 
		Successivamente, a Trieste, collaborò con il giornale\\
		 \textit{L’Indipendente}, di orientamento irredentista, così come lo scrittore. In seguito ad un investimento sbagliato del padre, Svevo conobbe la declassazione passando dall’agio della borghesia ad una condizione di ristrettezza. Egli fu costretto così a lavorare, trovò il suo impiego in una banca, dove rimase per circa 20 anni. Il lavoro impiegatizio era considerato da Svevo arido ed opprimente, l’unico canale di sfogo era per lui la letteratura; così nel 1892 pubblicò il suo primo romanzo “Una vita”, in cui rifletteva la sua esperienza lavorativa. Nel 1896 sposa la cugina Livia Veneziani di famiglia alto borghese, infatti Svevo lasci il suo impiego per lavorare nella ditta dei suoceri. Grazie a questo impiego Svevo passa da una condizione piccolo-borghese ad una alto-borghese e si trasforma in dirigente d’industria. In questo periodo lascia la letteratura considerandola come una perdita di tempo (scrisse anche il suo secondo romanzo Senilità (1892)). Negli anni precedenti al primo conflitto mondiale, Svevo conobbe James Joyce, con cui nacque una stretta amicizia; i due si scambiarono i propri scritti, Joyce “Dubliners” e Svevo i romanzi scritti in precedenza. Nel 1910 egli conobbe per la prima volta la psicoanalisi (cognato). La guerra diede il pretesto a Svevo per ricominciare l’attività letteraria, scrisse il suo terzo romanzo “La coscienza di Zeno” pubblicata nel 1923, che come gli altri due non ottenne inizialmente successo. L’autore così inviò il romanzo all’amico Joyce a Parigi, quest’ultimo lo sottopose ai critici francesi e ne ottenne una fama su vasta scala, prima in Francia poi in tutta Europa. L’unica persona in Italia che riconobbe il vero talento di Svevo fu Eugenio Montale. Muore nel 1928.
		
		\subsection{La coscienza di Zeno}
		La coscienza di Zeno è un memoriale, o confessione autobiografica, che il protagonista Zeno scrive sotto consiglio del suo psicanalista, il dottor S., a scopo terapeutico. Il memoriale viene unito ad un diario dove il protagonista spiega il perché ha abbandonato la cura, dichiarandosi guarito. Il libro è composto da capitoli tematici, che non seguono l’ordine cronologico degli eventi, e per narrarlo viene adottata la tecnica del tempo misto. Il protagonista è una figura di “inetto”, ovvero un incapace.
			\subsubsection{Il tempo misto}
			Tecnica tempo misto, teoria di Darwin, idee pessimistiche, inetto, Psicoanalisi funge da cornice (come la peste nel Decameron), uomo=malattia della terra, differenza animali – uomini, scienziati creano ordigni – qualcuno li deve usare, finale apocalittico, 
			ordigni (tutto quello creato dalla tecnica), complesso edipico, l’uomo possiede una parte inconscia che si manifesta attraverso sogni e libere associazioni.
	
	\chapter*{Le avanguardie}
		\section{Introduzione}
		\section[Marinetti]{I futuristi, Marinetti}
		Il principale esponente del movimento futurista fu Filippo Tommaso Marinetti. Egli pubblicò nel 1909 sul quotidiano “Le Figaro” il Manifesto del Futurismo, in cui formulava il suo programma di rivolta contro la cultura del passato, rimpiazzandola con valori come la velocità, il dinamismo e lo sfrenato attivismo. Il punto principale a cui si ispira questa corrente culturale è che la tecnica ha cambiato il mondo. Il futurismo inoltre esalta la violenza, il nazionalismo, la guerra come unica igiene del mondo, il disprezzo per la donna e la polemica alla sensibilità decadente, considerando quella letteratura ormai passata e quegli autori “passatisti”.
		Nel 1912 venne pubblicato un secondo manifesto, Il Manifesto tecnico della letteratura futurista, contenente lo stile che erano invitati a seguire i futuristi. Viene distrutta la sintassi, viene eliminata la punteggiatura, il testo deve visivamente sottolineare effetti particolari come un boato o un sussurro, per esempio cambiando il carattere; viene fatto largo uso di onomatopee, sinestesie, ma soprattutto analogie, diverse da quelle usate dai poeti precedenti. Le analogie usate dai futuristi legavano realtà diverse e lontanissime tra loro (es. uomo – torpediniere, donna – golfo, porta – rubinetto, piazza – imbuto). Marinetti quindi suggerisce il sostantivo-doppio e l’eliminazione dell’aggettivo, che porta ad un’inutile pensare. Alla distruzione della sintassi si accosta la teoria delle “parole in libertà” che consiste nel disporre i sostantivi “a caso, come nascono”.
		
		\subsection[Manifesto Tecnico]{Manifesto tecnico della letteratura futurista} Impatto Visivo
		
		\subsection[Manifesto Futurista]{Manifesto futurista}
		Il manifesto futurista: riassunto - DESCRIZIONE OGGETTIVA DEL MANIFESTO DEL MOVIMENTO FUTURISTA: Il 20 febbraio 1909 Filippo Tommaso Martinetti pubblica su “Le Figaro” il manifesto del futurismo che, in tredici punti, raccoglie le idee fondamentali del movimento. Il futurismo nasce quindi in epoca giolittiana, in un periodo di grande industrializzazione; il forte desiderio di rinnovamento spinge i futuristi ad appoggiare in un primo momento il fascismo. 
		
		Quando poi Mussolini deciderà di allearsi con Hitler, essi si allontaneranno sempre più dal regime. I futuristi vogliono esaltare “la bellezza della velocità”, “l'abitudine all'energia” e hanno come ideale l'uomo che usa la Terra come un'auto lanciata a tutta velocità “sul circuito della sua orbita”. Nello stesso tempo, però, mirano ad incendiare i musei, le biblioteche, le accademie, glorificano la guerra, che vedono come “unica igiene del mondo”, e la violenza. Martinetti afferma inoltre che solo le creazioni con un carattere aggressivo possono essere considerate delle vere opere d'arte.
		
		
		
		RIASSUNTO DEI 13 PUNTI DEL MANIFESTO ATTRAVERSO LA PARAFRASI DI ALCUNE LOCUZIONI 
		\begin{enumerate}
				
		\item 	“l'abitudine all'energia” = la consuetudine al dinamismo 
		\item “Il coraggio, l'audacia, la ribellione” = l'eroismo,	 l'ardimento, la rivolta 
		\item	“il passo di corsa, 
		 il salto mortale, lo schiaffo e il pugno” = la velocità, le acrobazie, la sberla, il cazzotto 
		\item	 “la bellezza della velocità” = il fascino della rapidità 
		\item “l'uomo che tiene il volante, la cui asta ideale attraversa la terra” = l'uomo, che usa la Terra, come un volante 
		\item “aumentare l'entusiastico fervore degli elementi primordiali” = accrescere il frenetico ardore dei principi originali 	
		\item	“Nessuna opera che non abbia un carattere aggressivo, può essere un capolavoro” = solo le creazioni di natura impetuosa possono essere di straordinaria bellezza 	\item“Il Tempo e lo Spazio morirono ieri” = I secoli e l'immensità svanirono nel passato	
		\item “sola igiene del mondo” = unico mezzo per purificare l'umanità 
		\item“vogliamo distruggere i musei, le biblioteche, le accademie di ogni specie” = abbiamo intenzione di far sparire tutti i luoghi dove si raccolgono opere d'arte, libri e tutte le associazioni di studenti
		\item “lune elettriche” = fonti di luce artificiali
		\item “fetida cancrena di professori” = corruzione abietta dei docenti 
		\item“mercato di rigattieri” = fiera di robivecchi
		
	\end{enumerate}
				
		

	\chapter*{Ermetismo}
		\section{Il movimento}
		L’Ermetismo è una corrente che nasce e Firenze e si sviluppa negli anni ’30 in Italia. L’origine può essere datata 1936, anno in cui Francesco Flora pubblica il libro “La Poesia ermetica” ma l’affermazione è decisa da “Letteratura come vita” di Carlo Bo, che contiene i fondamenti teorico-metodologici della poesia ermetica. 
		Il linguaggio è difficile e la poesia ermetica fa \textbf{uso consolidato del verso libero}, approfondisce la tematica esistenziale, ma soprattutto nei testi prevalgono le \textbf{motivazioni spirituali ed interiori}. 
		Il punto cruciale è che L’Ermetismo fa coincidere la poesia con la “\textbf{vita}”; la letteratura rappresenta la strada più completa per la conoscenza di noi stessi e si identifica con l’io profondo del soggetto proponendosi \textbf{ricerca della “verità”}.
		 Letteratura come vita infine significa \textbf{rifiuto della storia} (fascismo).
		Le conseguenze sono, un linguaggio difficile e al limite dell’incomunicabilità, l’espressione più privilegiata rimane l’analogia, e la poesia finisce per costituire la fonte privilegiata della conoscenza, la sola ed unica realtà.
		

		\section{Ungaretti}
		\subsection{Vita}
		
		Giuseppe Ungaretti nacque nel 1888 ad Alessandria d’Egitto. 
		In questa città studia e si avvicina ai maggiori scrittori moderni e passati tra cui Leopardi e Nietzsche. 
		Nel 1912 si trasferisce a Parigi dove approfondisce la conoscenza della poesia decadente e simbolista; inoltre frequenta gli ambienti dell’avanguardia. 
		Nel 1914 Ungaretti si spostò in Italia per arruolarsi come volontario e partire per la guerra. Fu inviato sul Carso, luogo in cui presero forma diverse sue poesie che successivamente furono raccolte nel volume “L’allegria”. 
		Alla fine del conflitto ritorna a Parigi; pochi anni dopo però si trasferì a Roma ed aderì a fascismo.
		Diventa presto uno dei più noti e prestigiosi intellettuali italiani, diventando un punto di riferimento per lo sviluppo della poesia ermetica; inoltre svolse l’attività di traduttore e di redattore.
		Successivamente diventa professore universitario, prima in Brasile poi a Roma. L’esperienza della Seconda Guerra Mondiale segna incisivamente i suoi componimenti di quel periodo. Prima di morire nel 1970 riesce a completare e pubblicare i suoi versi con il titolo “Vita d’un uomo”.
		
		
		\subsection{Poesie}
		
		\subsubsection{L'allegria}

		Il carattere principale dell’Allegria è l’autobiografia, intesa come interpretazione soggettiva dell’arte. 
		Infatti secondo Ungaretti vita e letteratura sono strettamente legate fra loro, quest’ultima svolge la funzione di svelare il lato nascosto delle cose. 
		Sul piano formale Ungaretti crea quella che viene chiamata “la poetica dell’essenziale”, che consiste in: un’estrema riduzione della frase, nell’abolizione quasi totale della punteggiatura, in un carattere visivo del testo (es spazi), nei versi liberi, e nell’uso dell’analogia, che si differenzia da quella precedente. 
		Grazie ad essa riesce a mettere in contatto immagini lontane (stati d’animo profondi) rapidamente e a rivelare la vera essenza della realtà.
		
		
		\renewcommand{\poemtoc}{subsection}
		\poemtitle{Soldati}
		\settowidth{\versewidth}{There was an old party of Lyme}
		
		\begin{verse}[\versewidth]
			Si sta come\\
			d'autunno\\
			sugli alberi\\
			le foglie\\
		\end{verse}
		
		\subparagraph[Soldati]{Spiegazione}Anche se la poesia è breve, Ungaretti riesce ad esprimere la condizione di soldato. Egli paragona infatti il soldato ad una foglia d'albero in autunno: basta un colpo di vento per far morire la foglia, così come basta un colpo di fucile a far cadere il soldato.
		

			\renewcommand{\poemtoc}{subsection}
		\poemtitle{Fratelli}
		\settowidth{\versewidth}{There was an old party of Lyme}
		
		\begin{verse}[\versewidth]
			Di che reggimento siete\\
			fratelli?\\
			
			Parola tremante\\
			nella notte\\
			
			Foglia appena nata\\
			
			Nell'aria spasimante\\
			involontaria rivolta\\
			dell'uomo presente alla sua\\
			fragilità\\
			
			Fratelli\\
		\end{verse}
			
		\subparagraph[Fratelli]{Spiegazione}Il tema principale è quindi quello della precarietà della vita, costantemente posta di fronte a una sensazione opprimente di morte. Anche in questi versi, come in Soldati, la fragilità umana è espressa dall'autore attraverso il confronto tra individuo e natura: i fratelli commilitoni diventano così “foglie appena nate” (v. 5). Con la definizione di “fratelli” (v. 10) i soldati riacquistano la propria umanità ed intima dignità. Attraverso l'immagine de l'“involontaria rivolta dell'uomo” (vv. 7-8), Ungaretti celebra l'istinto di quest'ultimo alla vita e il desiderio insito nell'animo di ognuno di sfuggire la morte e la guerra.
		
		\newpage
		\renewcommand{\poemtoc}{subsection}
		\poemtitle{Veglia}
		\settowidth{\versewidth}{There was an old party of Lyme}
		
		\begin{verse}[\versewidth]
			Un'intera nottata \\
			buttato vicino \\
			a un compagno \\
			massacrato \\
			con la sua bocca \\
			digrignata \\
			volta al plenilunio \\
			con la congestione \\
			delle sue mani \\
			penetrata \\
			nel mio silenzio \\
			ho scritto \\
			lettere piene d'amore \\
			
			Non sono mai stato \\
			tanto \\
			attaccato alla vita\\
		\end{verse}
		
		\subparagraph[Veglia]{Spiegazione}Fra Ungaretti e il soldato morto
		si crea un forte legame. Ai versi 8-10 leggiamo che il	gonfiore delle mani del soldato entra dentro al silenzio del
		poeta. Ungaretti vive insieme al compagno l’esperienza della morte, ma resiste alla morte e trova la forza di scrivere
		"lettere piene d’amore". Mentre vede da vicino la morte,Ungaretti si sente attaccato alla vita più che mai. Questo
		attaccamento alla vita, cioè questo forte desiderio di vivere, torna spesso nelle poesie dell’Allegria. Ecco, forse,
		perché Ungaretti sceglie questo titolo per una raccolta di poesie che parla della guerra e della morte: perché nota
		che spesso, proprio nelle situazioni più difficili, nascono nell’uomo una grande voglia di vivere e un’energia fortissima.
		

		\renewcommand{\poemtoc}{subsection}
		\poemtitle{Natale}
		\settowidth{\versewidth}{There was an old party of Lyme}
		
		\begin{verse}[\versewidth]
			Non ho voglia\\
			di tuffarmi\\
			in un gomitolo\\
			di strade\\
			
			Ho tanta\\
			stanchezza\\
			sulle spalle\\
			
			Lasciatemi così\\
			come una\\
			cosa\\
			posata\\
			in un\\
			angolo\\
			e dimenticata\\
			
			Qui\\
			non si sente\\
			altro\\
			che il caldo buono\\
			
			Sto\\
			con le quattro\\
			capriole\\
			di fumo\\
			del focolare\\
			
			Napoli, il 26 dicembre 1916
		\end{verse}
		
		\subparagraph[Natale]{Spiegazione}Natale è una poesia composta nel Natale del 1916, durante un periodo di licenza che Ungaretti trascorse a Napoli presso alcuni amici. Il momento di tregua dalla guerra consente al poeta un attimo di respiro: l'uomo straziato dal dolore per la morte dei compagni davanti ai suoi occhi.
		L'opera evidenzia tutta la tristezza del poeta, ancora impressionato dalla brutalità della guerra che non risparmia nessuno, e non ha voglia di passeggiare nelle piccole strade affollate di gente, durante le feste natalizie. Ungaretti frantuma i versi per dare l'impressione di un singhiozzo.Questo ritmo crea infatti tristezza e raggela l'animo del lettore, il che contrasta con l'immagine del caminetto, il quale più che calore pare evocare quelle emozioni che mancano.
		
		
		

		\renewcommand{\poemtoc}{subsection}
		\poemtitle{Sono una creatura}
		\settowidth{\versewidth}{There was an old party of Lyme}
		
		\begin{verse}[\versewidth]
			Come questa pietra\\
			del S. Michele\\
			così fredda\\
			così dura\\
			così prosciugata\\
			così refrattaria\\
			cos' totalmente \\
			disanimata\\
			
			Come questa pietra\\
			è il mio pianto\\
			che non si vede\\
			
			La morte\\
			si sconta\\
			vivendo\\
		\end{verse}
		
		\subparagraph[Sono una creatura]{Spiegazione}Il San Michele (luogo della poesia) è un monte del Carso ricordato per le sanguinose battaglie combattute durante la prima guerra mondiale, E' una zona aspra e arida. Formato da rocce porose, la pioggia non ha tempo a toccare terra che già viene assorbita dal terreno permeabilissimo. Simile all'acqua che viene immediatamente assorbita dal terreno è il pianto del poeta, un pianto senza lacrime, un dolore intimo che prosciuga l'anitna. Quasi come se la pace della morte si debba scontare con le sofferenze della vita.
		
		\renewcommand{\poemtoc}{subsection}
		\poemtitle{San Martino del carso}
		\settowidth{\versewidth}{There was an old party of Lyme}
		
		\begin{verse}[\versewidth]
		Di queste case\\
		Non è rimasto \\
		Che qualche\\
		Brandello di muro\\
		Di tanti\\
		Che mi corrispondevano\\
		Non è rimasto\\
		Neppure tanto\\
		Ma nel cuore\\
		Nessuna croce manca\\
		E’ il mio cuore\\
		Il paese più straziato\\
		\end{verse}
		
		\subparagraph[San Martino del carso]{Spiegazione} L’immagine di un paese distrutto dalla guerra, San Martino del Carso, è per il poeta l’equivalente delle distruzioni che sono celate nel suo cuore, causate dalla dolorosa perdita di tanti amici cari. Ancora una volta il poeta trova nelle immagini esterne una corrispondenza con quanto egli prova nei confronti dell’uomo, annullato dalla guerra. La lirica, di un’estrema essenzialità è tutta costruita su un gioco di rispondenze e di contrapposizioni sentimentali, ma anche verbali: di San Martino resta qualche brandello di muro, dei morti cari allo scrittore non resta nulla; San Martino è un paese straziato, più straziato è il cuore del poeta
		
		
		
		
		
		
	
		\section{Montale}
		\subsection{Vita}
		Eugenio Montale nasce a Genova nel 1896. Frequenta le scuole tecniche, si diploma come ragioniere e partecipa alla Prima Guerra Mondiale. Nel 1922 esordisce come poeta ed entra in contatto con l’antifascista Piero Gobetti. Scrive inoltre un omaggio a Italo Svevo. La prima raccolta di versi esce nel 1925 e s’intitola “Ossi di Seppia”; successivamente firma il manifesto degli intellettuali antifascisti, esponendo il suo dissenso politico e civile verso la dittatura. Si trasferisce a Firenze e pubblica la sua seconda raccolta “Le occasioni”, presso Einaudi; 
		intanto avvia anche un’intensa attività di traduttore. Successivamente si iscrisse al Partito d’Azione e fece parte del CLN toscano. Nel 1967 viene nominato senatore a vita e nel 1975 riceve il Nobel per la Letteratura; muore nel 1981.
		
		\subsection{Ossi di Seppia}
		Il libro è diviso in 4 sezioni: Movimenti, Ossi di Seppia, Mediterraneo, Meriggi e Ombre, che contiene i testi più complessi e ardui.
		Il testo è influenzato principalmente dal \textbf{ pessimismo di Schopenhauer} e dalla \textbf{ poesia Pascoliana}, sia per la scelta di trattare “oggetti poveri” sia per alcune scelte di stile, sia per i temi molto pessimistici.
		Uno dei temi centrali della raccolta è l’“arsura”, il paesaggio è quello ligure, caratterizzato dall’\textbf{aridità} e da un sole implacabile, che rappresenta una forza che inaridisce ogni forma di vita. L’aridità esterna quindi diviene anche inaridimento interiore, ovvero impossibilità di provare sentimenti vivi e intensi. Le uniche vie di fuga sono l’\textbf{indifferenza} e il \textbf{mare}. All’aridità si aggiunge un altro tema principale, il muro. Un \textbf{muro} invalicabile, insuperabile, che può essere solamente costeggiato, al di là del quale c’è la pienezza vitale e la verità. 
		Inoltre viene toccato il tema della \textbf{perdita dell’identità individuale}, carattere della cultura primonovecentesca.
		
		Tema dell'aridità: ambientato in Liguria
		Consapevolezza di tutto il cosmo (Leopardi)
		
		\subparagraph{Correlativo oggettivo}Sul piano dello stile Montale utilizza il correlativo oggettivo, una tecnica con la quale \textbf{associa oggetti a stati d’animo} del poeta es: rivo strozzato, foglia riarsa, cavallo stramazzato.
		
	\newpage
		\subsection{Poesie}
		
		\renewcommand{\poemtoc}{subsection}
		\poemtitle{Meriggiare pallido e assorto}
		\settowidth{\versewidth}{There was an old party of Lyme}
		
		\
		\begin{verse}[\versewidth]
			
			Meriggiare pallido e assorto\\
			presso un rovente muro d’orto,\\
			ascoltare tra i pruni e gli sterpi\\
			schiocchi di merli, frusci di serpi.\\
			Nelle crepe del suolo o su la veccia\\
			spiar le file di rosse formiche\\
			ch’ora si rompono ed ora s’intrecciano\\
			a sommo di minuscole biche.\\
			Osservare tra frondi il palpitare\\
			lontano di scaglie di mare\\
			mentre si levano tremuli scricchi\\
			di cicale dai calvi picchi.\\
			E andando nel sole che abbaglia\\
			sentire con triste meraviglia\\
			com’è tutta la vita e il suo travaglio\\
			in questo seguitare una muraglia\\
			che ha in cima cocci aguzzi di bottiglia.\\
		\end{verse}
	
	\subparagraph[Meriggiare pallido e assorto]{Spiegazione}La vita ha una componente noiosa come camminare fianco ad un muro e non poter passare oltre.
	
	\begin{enumerate}
		\item Aridità
		\item 	Inutilità della vita
		\item	Mare come via d'uscita\\
		
	\end{enumerate}
	
		\renewcommand{\poemtoc}{subsection}
		\poemtitle{Non chiederci la parola}
		\settowidth{\versewidth}{There was an old party of Lyme}
	
		\begin{verse}[\versewidth]
		
		Non chiederci la parola che squadri da ogni lato\\
		l'animo nostro informe, e a lettere di fuoco\\
		lo dichiari e risplenda come un croco\\
		perduto in mezzo a un polveroso prato. \\
		
		Ah l'uomo che se ne va sicuro,\\
		agli altri ed a se stesso amico,\\
		e l'ombra sua non cura che la canicola\\
		stampa sopra uno scalcinato muro!\\
		
		Non domandarci la formula che mondi possa aprirti,\\
		sì qualche storta sillaba e secca come un ramo.\\
		Codesto solo oggi possiamo dirti,\\
		ciò che non siamo, ciò che non vogliamo.\\
		
		\end{verse}
	
		\subparagraph[Non chiederci la parola]{Spiegazione}
	Vi sono tre macro concetti: non fateci domande sull'animo perchè solo i grandi poeti del passato come Petrarca e Leopardi sapevano descriverlo benissimo.
	Esistono persone che non si guardano attorno ed hanno una visione superficiale della vita ad esempio i conformisti e coloro che si sentono sempre sicuri.
	Non abbiamo soluzioni ma conosciamo bene i lati negativi della vita.\\
	
	\twocolumn
		\renewcommand{\poemtoc}{subsection}
		\poemtitle{Spesso il male di vivere}
		\settowidth{\versewidth}{There was an old party of Lyme}
	
		\begin{verse}[\versewidth]
		Spesso il male di vivere ho incontrato	\\	
		era il rivo strozzato che gorgoglia		\\
		era l'incartocciarsi della foglia		\\
		riarsa, era il cavallo stramazzato.\\		
		
		
		Bene non seppi, fuori del prodigio\\
		che schiude la divina Indifferenza:\\
		era la statua nella sonnolenza\\
		del meriggio, e la nuvola, e il falco alto levato.\\
		\end{verse}
	
	\subparagraph[Spesso il male di vivere]{Spiegazione}
	Il concetto della vita è in correlazione con oggetti, inizia subito dicendo che il bene non esiste ma una forma neutra di "non male".\\
	
		
		
		
		
		
		
		
		
		
		
		
		
		
		
		
		
	
		\section{Quasimodo}
		
		\subsection{Vita}
			Quasimodo nasce a Modica nel 1901 e a 19 anni si stabilisce a Roma, iniziando a studiare le lingue classiche. È riconosciuto come uno dei più significativi esponenti dell’Ermetismo, infatti il suo stile è molto complicato. C’è una netta separazione dalla lingua parlata, inoltre si nota la frequenza delle analogie e la confusione nei rapporti logici tra i periodi. Successivamente alla Seconda Guerra Mondiale la poesia di Quasimodo diventa testimonianza di politica e di polemica sociale (poesia impegnata). I temi trattano una realtà più concreta aprendosi a forme di messaggio più accessibili e comunicative. 
			La vita letteraria di Quasimodo si divide quindi fra Ermetismo e Realismo. Nel 1960 vinse il premio Nobel per la Letteratura; muore a Napoli nel 1968.
		
	
		\subsection{Poesie}
		
		\renewcommand{\poemtoc}{subsection}
		\poemtitle{Ed è subito sera}
		\settowidth{\versewidth}{There was an old party of Lyme}
		
		\begin{verse}[\versewidth]
			Ognuno sta solo sul cuor della terra\\
			trafitto da un raggio di sole:\\
			ed è subito sera. \\
		\end{verse}
		
		\subparagraph[Ed è subito sera]{Spiegazione}
		In questa poesia il poeta ha racchiuso i tre momenti della vita dell'uomo: la solitudine, derivata dall'incomunicabilità; l'alternarsi della gioia e del dolore; il senso della precarietà della vita. Ognuno, dice il poeta, pur vivendo in mezzo agli uomini (sul cuor della terra) si sente fortemente solo a causa dell'impossibilità di stabilire un rapporto duraturo con qualcuno. L'ipotesi più accreditata del significato di star solo "sul cuor della terra" attribuisce alle parole il significato di star solo nel momento individuale ed intimo della ricerca del senso dell'esistenza, ovvero di ciò che permette all'uomo di sorpassare la morte. Tuttavia, pur essendo solo, viene stimolato dalle illusioni (un raggio di sole), dalla ricerca di una felicità a volte apparente. Questa ricerca è nello stesso tempo gioia e dolore, perciò il poeta usa il termine "trafitto", cioè, ferito dal raggio di sole stesso. E intanto, come alla luce del giorno succede rapidamente l'oscurità notturna, per la vita dell'uomo giunge la morte: ed è subito sera.
	
		\renewcommand{\poemtoc}{subsection}
		\poemtitle{Alle fronde dei salici}
		\settowidth{\versewidth}{There was an old party of Lyme}
		\begin{verse}[\versewidth]
			E come potevamo noi cantare\\
			con il piede straniero sopra il cuore,\\
			fra i morti abbandonati nelle piazze\\
			sull’erba dura di ghiaccio, al lamento\\
			d’agnello dei fanciulli, all’urlo nero\\
			della madre che andava incontro al figlio\\
			crocifisso sul palo del telegrafo?\\
			Alle fronde dei salici, per voto,\\
			anche le nostre cetre erano appese,\\
			oscillavano lievi al triste vento.\\
		\end{verse}
	
			\subparagraph[Alle fronte dei salici]{Spiegazione}
		Il testo consta di due parti. Nella prima il poeta presenta una serie di figure che esprimono il dolore provato per l'invasione nazista dell'Italia, mentre nella seconda esprime la sua opinione sulla poesia ed il suo impegno davanti a tanto dolore.
		Quasimodo appare qui orientato verso una poesia impegnata civilmente; ha sostituito all’esperienza individuale l’esperienza collettiva, alla parola simbolo un linguaggio più aperto, disteso, colloquiale. Testimone degli orrori della seconda guerra mondiale, in particolare del massacro dei civili sotto i bombardamenti e della violenza nazifascista, il poeta si sente partecipe della sofferenza di tutti gli uomini e, spinto dalla pietà, cerca di ricomporre i frammenti di un’umanità e di una civiltà offese e distrutte dalla violenza, nella speranza che dopo tanto orrore possa nascere un uomo nuovo capace di trasformare il mondo.
		
		\textbf{Figure retoriche:}
		
		
		\begin{enumerate}
			\item urlo nero: sinestesia
			\item nostre cetre: metafora
			\item triste vento: sinestesia
			\item analogia tra il pianto dei bambini e il belato dell'agnello, animale simbolico che rappresenta l'innocenza e il sacrificio.

		\end{enumerate}
	
	\twocolumn
		\renewcommand{\poemtoc}{subsection}
		\poemtitle{Milano, agosto 1943}
		\settowidth{\versewidth}{There was an old party of Lyme}
		\begin{verse}[\versewidth]
		Invano cerchi tra la polvere,\\
		povera mano, la città è morta.\\
		È morta: s’è udito l’ultimo rombo\\
		sul cuore del Naviglio. E l’usignolo\\
		è caduto dall’antenna, alta sul convento,\\
		dove cantava prima del tramonto.\\
		Non scavate pozzi nei cortili:\\
		i vivi non hanno più sete.\\
		Non toccate i morti, così rossi, così gonfi:\\
		lasciateli nella terra delle loro case:\\
		la città è morta, è morta.\\
	\end{verse}


		\subparagraph[Milano, agosto 1943]{Spiegazione}
		Milano, agosto 1943, fa parte della raccolta Giorno dopo giorno (1947) e rientra nella produzione poetica di impegno civile di Quasimodo, ispirata alle vicende della seconda guerra mondiale. 
		Il poeta abbandona i temi dell’ermetismo, pur mantenendo gli stessi modi espressivi, per occuparsi di tematiche ispirate alla realtà storica, politica e sociale del tempo. 
		Questa poesia rievoca i bombardamenti che colpirono Milano nell’agosto del 1943, seminando distruzione e morte e spegnendo nei sopravvissuti ogni attaccamento alla vita. Il poeta esprime lo sgomento e l’incredulità di fronte a tale scempio che ha reso la città un grande e desolato cimitero. Di fronte a tale violenza è inutile scavare nelle macerie, tentare di riprendere la vita di tutti i giorni (scavare i pozzi per la sete dei superstiti), seppellire i morti: la città è morta e con essa anche la voglia di vivere degli uomini.
		
		Ricordiamo la \textbf{forma colloquiale} e l'\textbf{uso del "tu"}.
	
	\chapter*{Letteratura Americana}
	
		\section{Pavese}
		\subsection[Vita]{Vita}
		Cesare Pavese nasce a Santo Stefano Belbo nel 1908 nelle langhe cuneesi. Lo scrittore è introverso e la difficile situazione famigliare spiega la sua fragilità psicologica (“il mestiere di vivere”). Già dall’adolescenza manifesta la difficoltà di inserirsi nella vita cittadina; quando è in campagna si comporta da cittadino, in città da contadino. 
		Al liceo sviluppa un’avversione nei confronti del fascismo, che manifesta \\nell’interessa alla letteratura americana, diversa dalle imposizioni del regime. Lavora così per la casa editrice Einaudi (come Vittorini); una delle sue traduzioni più riuscite è Moby Dick. Nel 1935 venne trovato in possesso di alcune lettere compromettenti e fu condannato a 3 anni di confino in Calabria, dove scrisse “Paesi tuoi”, e al termine del quale tornò a Torino, riprendendo in seguito gli impegni editoriali. Durante la guerra non partecipò alla resistenza partigiana; nel dopoguerra però aderì al Partito Comunista. Nel 1950 a causa della sua fragilità psicologica, si suicidò.
		Pavese è uno scrittore neorealista che racconta della vita contadina, e ambienta i suoi testi principalmente in campagna, nelle Langhe piemontesi.  
		
		\subsection[Opere]{Opere}	
		Il suo primo romanzo “Paesi tuoi” (1941) parla di un uomo, il protagonista Berto, rappresentante della civiltà moderna ed industriale, che viene catapultato in una realtà non sua, quella contadina. La campagna si rivela essere un mondo selvaggio e tutt’altro che tranquillo; in conclusione il protagonista non si riconosce più in alcuna realtà, né quella cittadina, né quella dei campi. In questo libro Pavese reintroduce il racconto a dialoghi, e riprende la letteratura americana sul piano della “crudezza”.
		
		\newpage
		\section{Vittorini}

		\subsection{Vita}
		Elio Vittorini nasce a Siracusa ne 1908 da famiglia modesta; fu un autodidatta attraverso molte letture e rapporti intensi con il mondo letterario dell’epoca. Negli anni giovanili il suo orientamento politico, era verso il “fascismo di sinistra”. Ma la guerra di Spagna aprì gli occhi allo scrittore sulla vera natura del fascismo, di conseguenza partecipò ad un’attività clandestina antifascista. Nel 1939 si trasferì a Milano, dedicandosi insieme a Pavese, alla traduzione e alla diffusione della letteratura americana, che introduceva un nuovo stile di scrittura (molti dialoghi, linguaggio molto “crudo”). Durante la Seconda Guerra Mondiale entra nel Partito Comunista e partecipa alla Resistenza. Una volta concluso in conflitto fonda la rivista “Il Politecnico”, con il quale puntava a diffondere la cultura, credendo che, rendendo gli uomini più colti, si potessero evitare gli errori del passato (es guerra). In questo periodo lo scrittore si scontrò con il leader comunista Togliatti, che voleva Vittorini più vicino alle idee comuniste, ma l’autore pensava che l’unico modo per sviluppare la cultura fosse libero da qualsiasi idea politica. Nel 1947, privato dai finanziamenti del partito, chiuse l’esperienza de “Il Politecnico”; successivamente lavorò per la casa editrice Einaudi, dirigendo la collana dei “Gettoni”. Morì nel 1966.
		
		\subsection[Uomini e no]{Uomini e no}
		Il testo principale di cui abbiamo letto una parte è Uomini e no. Il libro parla della Resistenza in città; il brano che abbiamo tratto dal libro è ambientato a Milano, durante la Resistenza, e parla di un ufficiale nazista che fa sbranare un venditore ambulante, colpevole di aver ucciso precedentemente la sua cagna Greta. Nella parte finale del brano i sottoposti discutono se sia stato giusto o meno l’omicidio. Inoltre c’è un commento dell’autore, scritto in corsivo, in cui si chiede appunto se quelle azioni siano riconducibili a degli uomini.
		
	
	\chapter*{Neorealisti}
		\section{Cinema e Letteratura}
		Il Neorealismo è una corrente, sviluppatasi negli anni ’30 in Italia, che si fa carico di descrivere gli orrori della guerra, l’esperienza della Resistenza, i problemi della ricostruzione, e le ingiustizie sociali (letteratura impegnata), al fine di rappresentare la vita nei suoi aspetti più umani. Il genere letterario che si fa banditore di questa corrente è il romanzo, grazie alle sue capacità analitiche. Un’influenza fondamentale fu apportata dalla letteratura americana, tradotta e ripresa negli anni ’30 soprattutto da Pavese e Vittorini.\\
		
		\textbf{Cinema:}		
		\begin{enumerate}
			\item De Sica (Ladri di biciclette, Sciuscià)
			\item Rossellini (Roma città aperta)						
		\end{enumerate}
	
		\textbf{Letteratura:}	
		\begin{enumerate}
			\item Moravia
			\item Pratolini	
			\item Cassola
			\item Vittorini				
		\end{enumerate}
		
	

		\section{Italo Calvino}
			\subsection{Vita}		
			Calvino nacque nel 1923 a Santiago di Las Vegas, a cuba, dove il padre, noto agronomo, dirigeva una stazione sperimentale di agricoltura.
			Nel 1925 la famiglia si trasferì a Sanremo dove ebbe un'educazione laica e un forte interesse per le scienze tanto che si iscrisse alla facoltà di agraria.
			
			Nel 1943 per evitare l'arruolamento  entrò nella resistenza e nel dopoguerra, militante del PCI, passò alla facoltà di Lettere di Torino dove ebbe i primi contatti con la casa editrice Einaudi entrando in contatto con Pavese e Vittorini.
			
			Nel 1956, dopo l'invasione Sovietica dell'Ungheria, si allontanò dal PCI.
			
			\subsection{Neorealismo e componente fantastica}
			Il primo romanzo, "Il sentiero dei nidi di ragno", è un'opera neorealista dove Calvino esprime il fervore degli anni postbellici.
			Tuttavia il suo scopo non è celebrare la resistenza ma \textbf{dimostrare che anche chi era impegnato nella lotta senza chiare motivazioni ideali prendeva parte attiva nella storia}. La banda partigiana che rappresenta infatti è costituita dagli scarti di tutte le altre formazioni e da balordi. 
			Si manifesta in questo modo \textbf{l'indipendenza intellettuale} che lo contraddistinguerà.
			Calvino si allontana dagli standard neorealisti poichè le sue opere non hanno un intento documentaristico ma sono trasferite in un clima di \textbf{fiaba}.
			
			Sempre nel "i sentieri dei nidi di ragno" si manifesta l'estraneità dello sguardo del bambino e si metaforizza il suo stesso rapporto con la guerra partigiana, l'inferiorità sentita come borghese rispetto a quel mondo.
					
			\chapter*{Schemi}
			
			
					
						\begin{table}[]
				\centering
				\caption{Date di Nascita autori}
				\label{Date di Nascita autori}
				\begin{tabular}{lll}
					Autore      & Anno nascita & Luogo                   \\
					Pascoli     & 1855         & SanSan Mauro in Romagna \\
					Italo Svevo & 1861         & Trieste                 \\
					D'annunzio  & 1863         & Pescara                 \\
					Pirandello  & 1867         & Agrigento               \\
					Marinetti   & 1876         & Alessandria d'Egitto    \\
					Joyce       & 1882         & Dublino                 \\
					Kafka       & 1883         & Praga                   \\
					Ungaretti   & 1888         & Alessandria d'Egitto    \\
					Montale     & 1896         & Genova                  \\
					Quasimodo   & 1901         & Sicilia                 \\
					Vittorini   & 1908         & Siracusa                \\
					Pavese      & 1908         & Langhe Piemontesi       \\
					Calvino     & 1923         & Cuba                   
				\end{tabular}
			\end{table}
			
			
	
	

	
	
	
	
	
	
	
	
\end{document}