\documentclass[12pt]{book}
\usepackage[utf8]{inputenc}
\title{La Signora Frola e il Signor Ponza, suo genero}
\author{Luigi Pirandello}
\begin{document}

\maketitle
Ma insomma, ve lo figurate? c’è da ammattire sul serio tutti quanti a non poter sapere chi tra i due sia il pazzo, se questa signora Frola o questo signor Ponza, suo genero. Cose che càpitano soltanto a Valdana, città disgraziata, calamìta di tutti i forestieri eccentrici!

Pazza lei o pazzo lui; non c’è via di mezzo: uno dei due dev’esser pazzo per forza. Perché si tratta niente meno che di questo... Ma no, è meglio esporre prima con ordine.

Sono, vi giuro, seriamente costernato dell’angoscia in cui vivono da tre mesi gli abitanti di Valdana, e poco m’importa della signora Frola e del signor Ponza, suo genero. Perché, se è vero che una grave sciagura è loro toccata, non è men vero che uno dei due, almeno, ha avuto la fortuna d’impazzirne e l’altro l’ha ajutato, séguita ad ajutarlo così che non si riesce, ripeto, a sapere quale dei due veramente sia pazzo; e certo una consolazione meglio di questa non se la potevano dare.

 Ma dico di tenere così, sotto quest’incubo, un’intera cittadinanza, vi par poco? togliendole ogni sostegno al giudizio, per modo che non possa più distinguere tra fantasma e realtà. Un’angoscia, un perpetuo sgomento. 
 
 Ciascuno si vede davanti, ogni giorno, quei due; li guarda in faccia; sa che uno dei due è pazzo; li studia, li squadra, li spia e, niente! non poter scoprire quale dei due; dove sia il fantasma, dove la realtà. 
 
 Naturalmente, nasce in ciascuno il sospetto pernicioso che tanto vale allora la realtà quanto il fantasma, e che ogni realtà può benissimo essere un fantasma e viceversa. 
 
 Vi par poco? Nei panni del signor prefetto, io darei senz’altro, per la salute dell’anima degli abitanti di Valdana, lo sfratto alla signora Frola e al signor Ponza, suo genero.

Ma procediamo con ordine.

Questo signor Ponza arrivò a Valdana or sono tre mesi, segretario di prefettura. Prese alloggio nel casolare nuovo all’uscita del paese, quello che chiamano "il Favo". Lì.

 All’ultimo piano, un quartierino. Tre finestre che danno sulla campagna, alte, tristi (ché la facciata di là, all’aria di tramontana, su tutti quegli orti pallidi, chi sa perché, benché nuova, s’è tanto intristita) e tre finestre interne, di qua, sul cortile, ove gira la ringhiera del ballatojo diviso da tramezzi a grate. Pendono da quella ringhiera, lassù lassù, tanti panierini pronti a esser calati col cordino a un bisogno.
 
Nello stesso tempo, però, con maraviglia di tutti, il signor Ponza fissò nel centro della città, e propriamente in Via dei Santi n. 15, un altro quartierino mobigliato di tre camere e cucina. Disse che doveva servire per la suocera, signora Frola.

 E difatti questa arrivò cinque o sei giorni dopo; e il signor Ponza si recò ad accoglierla, lui solo, alla stazione e la condusse e la lasciò lì, sola.

Ora, via, si capisce che una figliuola, maritandosi, lasci la casa della madre per andare a convivere col marito, anche in un’altra città; ma che questa madre poi, non reggendo a star lontana dalla figliuola, lasci il suo paese, la sua casa, e la segua, e che nella città dove tanto la figliuola quanto lei sono forestiere vada ad abitare in una casa a parte, questo non si capisce più facilmente; o si deve ammettere tra suocera e genero una così forte incompatibilità da rendere proprio impossibile la convivenza, anche in queste condizioni.

Naturalmente a Valdana dapprima si pensò così. E certo chi scapitò per questo nell’opinione di tutti fu il signor Ponza. Della signora Frola, se qualcuno ammise che forse doveva averci anche lei un po’ di colpa, o per scarso compatimento o per qualche caparbietà o intolleranza, tutti considerarono l’amore materno che la traeva appresso alla figliuola, pur condannata a non poterle vivere accanto.


Gran parte ebbe in questa considerazione per la signora Frola e nel concetto che subito del signor Ponza s’impresse nell’animo di tutti, che fosse cioè duro, anzi crudele, anche l’aspetto dei due, bisogna dirlo.

 Tozzo, senza collo, nero come un africano, con folti capelli ispidi su la fronte bassa, dense e aspre sopracciglia giunte, grossi mustacchi lucidi da questurino, e negli occhi cupi, fissi, quasi senza bianco, un’intensità violenta, esasperata, a stento contenuta, non si sa se di doglia tetra o di dispetto della vista altrui, il signor Ponza non è fatto certamente per conciliarsi la simpatia o la confidenza.
 
  Vecchina gracile, pallida, è invece la signora Frola, dai lineamenti fini, nobilissimi, e una aria malinconica, ma d’una malinconia senza peso, vaga e gentile, che non esclude l’affabilità con tutti.

Ora di questa affabilità, naturalissima in lei, la signora Frola ha dato subito prova in città, e subito per essa nell’animo di tutti è cresciuta l’avversione per il signor Ponza; giacché chiaramente è apparsa a ognuno l’indole di lei, non solo mite, remissiva, tollerante, ma anche piena d’indulgente compatimento per il male che il genero le fa; e anche perché s’è venuto a sapere che non basta al signor Ponza relegare in una casa a parte quella povera madre, ma spinge la crudeltà fino a vietarle anche la vista della figliuola.

Se non che, non crudeltà, protesta subito nelle sue visite alle signore di Valdana la signora Frola, ponendo le manine avanti, veramente afflitta che si possa pensare questo di suo genero.

 E s’affretta a decantarne tutte le virtù, a dirne tutto il bene possibile e immaginabile; quale amore, quante cure, quali attenzioni egli abbia per la figliuola, non solo, ma anche per lei, sì, sì, anche per lei; premuroso, disinteressato... Ah, non crudele, no, per carità! C’è solo questo: che vuole tutta, tutta per sé la mogliettina, il signor Ponza, fino al punto che anche l’amore, che questa deve avere (e l’ammette, come no?) per la sua mamma, vuole che le arrivi non direttamente, ma attraverso lui, per mezzo di lui, ecco.
 
  Sì, può parere crudeltà, questa, ma non lo è; è un’altra cosa, un’altra cosa ch’ella, la signora Frola, intende benissimo e si strugge di non sapere esprimere. Natura, ecco... ma no, forse una specie di malattia... come dire? Mio Dio, basta guardarlo negli occhi. Fanno in prima una brutta impressione, forse, quegli occhi; ma dicono tutto a chi, come lei, sappia leggere in essi: la pienezza chiusa, dicono, di tutto un mondo d’amore in lui, nel quale la moglie deve vivere senza mai uscirne minimamente, e nel quale nessun altro, neppure la madre, deve entrare. Gelosia? Sì, forse; ma a voler definire volgarmente questa totalità esclusiva d’amore.
  
Egoismo? Ma un egoismo che si dà tutto, come un mondo, alla propria donna! Egoismo, in fondo, sarebbe quello di lei a voler forzare questo mondo chiuso d’amore, a volervisi introdurre per forza, quand’ella sa che la figliuola è felice, così adorata... Questo a una madre può bastare! Del resto, non è mica vero ch’ella non la veda, la sua figliuola. Due o tre volte al giorno la vede: entra nel cortile della casa; suona il campanello e subito la sua figliuola s’affaccia di lassù.

– Come stai Tildina?

– Benissimo, mamma. Tu?

– Come Dio vuole, figliuola mia. Giù, giù il panierino!

E nel panierino, sempre due parole di lettera, con le notizie della giornata. Ecco, le basta questo. Dura ormai da quattr’anni questa vita, e ci s’è abituata la signora Frola. Rassegnata, sì. E quasi non ne soffre più.

Com’è facile intendere, questa rassegnazione della signora Frola, quest’abitudine ch’ella dice d’aver fatto al suo martirio, ridondano a carico del signor Ponza, suo genero, tanto più, quanto più ella col suo lungo discorso si affanna a scusarlo.

Con vera indignazione perciò, e anche dirò con paura, le signore di Valdana che hanno ricevuto la prima visita della signora Frola, accolgono il giorno dopo l’annunzio di un’altra visita inattesa, del signor Ponza, che le prega di concedergli due soli minuti d’udienza, per una "doverosa dichiarazione", se non reca loro incomodo.

Affocato in volto, quasi congestionato, con gli occhi più duri e più tetri che mai, un fazzoletto in mano che stride per la sua bianchezza, insieme coi polsini e il colletto della camicia, sul nero della carnagione, del pelame e del vestito, il signor Ponza, asciugandosi di continuo il sudore che gli sgocciola dalla fronte bassa e dalle gote raschiose e violacee, non già per il caldo, ma per la violenza evidentissima dello sforzo che fa su se stesso e per cui anche le grosse mani dalle unghie lunghe gli tremano; in questo e in quel salotto, davanti a quelle signore che lo mirano quasi atterrite, domanda prima se la signora Frola, sua suocera, è stata a visita da loro il giorno avanti; poi, con pena, con sforzo, con agitazione di punto in punto crescenti, se ella ha parlato loro della figliuola e se ha detto che egli le vieta assolutamente di vederla e di salire in casa sua.

Le signore, nel vederlo così agitato, com’è facile immaginare, s’affrettano a rispondergli che la signora Frola, sì, è vero, ha detto loro di quella proibizione di vedere la figlia, ma anche tutto il bene possibile e immaginabile di lui, fino a scusarlo, non solo, ma anche a non dargli nessun’ombra di colpa per quella proibizione stessa.

Se non che, invece di quietarsi, a questa risposta delle signore, il signor Ponza si agita di più; gli occhi gli diventano più duri, più fissi, più tetri; le grosse gocce di sudore più spesse; e alla fine, facendo uno sforzo ancor più violento su se stesso, viene alla sua "dichiarazione doverosa".

La quale è questa, semplicemente: che la signora Frola, poveretta, non pare, ma è pazza.

Pazza da quattro anni, sì. E la sua pazzia consiste appunto nel credere che egli non voglia farle vedere la figliuola. Quale figliuola? È morta, è morta da quattro anni la figliuola: e la signora Frola, appunto per il dolore di questa morte, è impazzita: per fortuna, impazzita, sì, giacché la pazzia è stata per lei lo scampo dal suo disperato dolore. Naturalmente non poteva scamparne, se non così, cioè credendo che non sia vero che la sua figliuola è morta e che sia lui, invece, suo genero, che non vuole più fargliela vedere.

Per puro dovere di carità verso un’infelice, egli, il signor Ponza, seconda da quattro anni, a costo di molti e gravi sacrifici, questa pietosa follia: tiene, con dispendio superiore alle sue forze, due case: una per sé, una per lei; e obbliga la sua seconda moglie, che per fortuna caritatevolmente si presta volentieri, a secondare anche lei questa follia.

 Ma carità, dovere, ecco, fino a un certo punto: anche per la sua qualità di pubblico funzionario, il signor Ponza non può permettere che si creda di lui, in città, questa cosa crudele e inverosimile: ch’egli cioè, per gelosia o per altro, vieti a una povera madre di vedere la propria figliuola.


Dichiarato questo, il signor Ponza s’inchina innanzi allo sbalordimento delle signore, e va via. Ma questo sbalordimento delle signore non ha neppure il tempo di scemare un po’, che rieccoti la signora Frola con la sua aria dolce di vaga malinconia a domandare scusa se, per causa sua, le buone signore si sono prese qualche spavento per la visita del signor Ponza, suo genero.

E la signora Frola, con la maggior semplicità e naturalezza del mondo, dichiara a sua volta, ma in gran confidenza, per carità! poiché il signor Ponza è un pubblico funzionario, e appunto per questo ella la prima volta s’è astenuta dal dirlo, ma sì, perché questo potrebbe seriamente pregiudicarlo nella carriera; il signor Ponza, poveretto – ottimo, ottimo inappuntabile segretario alla prefettura, compìto, preciso in tutti i suoi atti, in tutti i suoi pensieri, pieno di tante buone qualità – il signor Ponza, poveretto, su quest’unico punto non... non ragiona più, ecco; il pazzo è lui, poveretto; e la sua pazzia consiste appunto in questo: nel credere che sua moglie sia morta da quattro anni e nell’andar dicendo che la pazza è lei, la signora Frola che crede ancora viva la figliuola.

 No, non lo fa per contestare in certo qual modo innanzi agli altri quella sua gelosia quasi maniaca e quella crudele proibizione a lei di vedere la figliuola, no; crede, crede sul serio il poveretto che sua moglie sia morta e che questa che ha con sé sia una seconda moglie.
 
  Caso pietosissimo! Perché veramente col suo troppo amore quest’uomo rischiò in prima di distruggere, d’uccidere la giovane moglietta delicatina, tanto che si dovette sottrargliela di nascosto e chiuderla a insaputa di lui in una casa di salute.

 Ebbene, il povero uomo, a cui già per quella frenesia d’amore s’era anche gravemente alterato il cervello, ne impazzì; credette che la moglie fosse morta davvero: e questa idea gli si fissò talmente nel cervello, che non ci fu più verso di levargliela, neppure quando, ritornata dopo circa un anno florida come prima, la moglietta gli fu ripresentata. 
 
 La credette un’altra; tanto che si dovette con l’ajuto di tutti, parenti e amici, simulare un secondo matrimonio, che gli ha ridato pienamente l’equilibrio delle facoltà mentali.

Ora la signora Frola crede d’aver qualche ragione di sospettare che da un pezzo suo genero sia del tutto rientrato in sé e ch’egli finga, finga soltanto di credere che sua moglie sia una seconda moglie, per tenersela così tutta per sé, senza contatto con nessuno, perché forse tuttavia di tanto in tanto gli balena la paura che di nuovo gli possa esser sottratta nascostamente. 

Ma sì. Come spiegare, se no, tutte le cure, le premure che ha per lei, sua suocera, se veramente egli crede che è una seconda moglie quella che ha con sé? Non dovrebbe sentire l’obbligo di tanti riguardi per una che, di fatto, non sarebbe più sua suocera, è vero? Questo, si badi, la signora Frola lo dice, non per dimostrare ancor meglio che il pazzo è lui; ma per provare anche a se stessa che il suo sospetto è fondato.

– E intanto, – conclude con un sospiro che su le labbra le s’atteggia in un dolce mestissimo sorriso, – intanto la povera figliuola mia deve fingere di non esser lei, ma un’altra, e anch’io sono obbligata a fingermi pazza credendo che la mia figliuola sia ancora viva. 

Mi costa poco, grazie a Dio, perché è là, la mia figliuola, sana e piena di vita; la vedo, le parlo; ma sono condannata a non poter convivere con lei, e anche a vederla e a parlarle da lontano, perché egli possa credere, o fingere di credere che la mia figliuola, Dio liberi, è morta e che questa che ha con sé è una seconda moglie. 

Ma torno a dire, che importa se con questo siamo riusciti a ridare la pace a tutti e due? So che la mia figliuola è adorata, contenta; la vedo; le parlo; e mi rassegno per amore di lei e di lui a vivere così e a passare anche per pazza, signora mia, pazienza...

Dico, non vi sembra che a Valdana ci sia proprio da restare a bocca aperta, a guardarci tutti negli occhi, come insensati? A chi credere dei due? Chi è il pazzo? Dov’è la realtà? dove il fantasma?

Lo potrebbe dire la moglie del signor Ponza. Ma non c’è da fidarsi se, davanti a lui, costei dice d’esser seconda moglie; come non c’è da fidarsi se, davanti alla signora Frola, conferma d’esserne la figliuola. Si dovrebbe prenderla a parte e farle dire a quattr’occhi la verità. Non è possibile.

 Il signor Ponza – sia o no lui il pazzo – è realmente gelosissimo e non lascia vedere la moglie a nessuno. La tiene lassù, come in prigione, sotto chiave; e questo fatto è senza dubbio in favore della signora Frola; ma il signor Ponza dice che è costretto a far così, e che sua moglie stessa anzi glielo impone, per paura che la signora Frola non le entri in casa all’improvviso.

 Può essere una scusa. Sta anche di fatto che il signor Ponza non tiene neanche una serva in casa. Dice che lo fa per risparmio, obbligato com’è a pagar l’affitto di due case; e si sobbarca intanto a farsi da sé la spesa giornaliera, e la moglie, che a suo dire non è la figlia della signora Frola, si sobbarca anche lei per pietà di questa, cioè d’una povera vecchia che fu suocera di suo marito, a badare a tutte le faccende di casa, anche alle più umili, privandosi dell’ajuto di una serva. Sembra a tutti un po’ troppo. 

Ma è anche vero che questo stato di cose, se non con la pietà, può spiegarsi con la gelosia di lui.

Intanto, il signor Prefetto di Valdana s’è contentato della dichiarazione del signor Ponza. Ma certo l’aspetto e in gran parte la condotta di costui non depongono in suo favore, almeno per le signore di Valdana più propense tutte quante a prestar fede alla signora Frola. 

Questa, difatti, viene premurosa a mostrar loro le letterine affettuose che le cala giù col panierino la figliuola, e anche tant’altri privati documenti, a cui però il signor Ponza toglie ogni credito, dicendo che le sono stati rilasciati per confortare il pietoso inganno.

Certo è questo, a ogni modo: che dimostrano tutt’e due, l’uno per l’altra, un meraviglioso spirito di sacrifizio, commoventissimo; e che ciascuno ha per la presunta pazzia dell’altro la considerazione più squisitamente pietosa. 

Ragionano tutt’e due a meraviglia; tanto che a Valdana non sarebbe mai venuto in mente a nessuno di dire che l’uno dei due era pazzo, se non l’avessero detto loro: il signor Ponza della signora Frola, e la signora Frola del signor Ponza.

La signora Frola va spesso a trovare il genero alla prefettura per aver da lui qualche consiglio, o lo aspetta all’uscita per farsi accompagnare in qualche compera: e spessissimo, dal canto suo, nelle ore libere e ogni sera il signor Ponza va a trovare la signora Frola nel quartierino mobigliato; e ogni qual volta per caso l’uno s’imbatte nell’altra per via, subito con la massima cordialità si mettono insieme; egli le dà la destra e, se stanca, le porge il braccio, e vanno così, insieme, tra il dispetto aggrondato e lo stupore e la costernazione della gente che li studia, li squadra, li spia e, niente!, non riesce ancora in nessun modo a comprendere quale sia il pazzo dei due, dove sia il fantasma, dove la realtà.

\end{document}